\thispagestyle{plain}
\chapter*{\centering \begin{huge}Tóm tắt\end{huge}}

\noindent
%\large
Bài toán phát hiện bất thường bằng phương pháp dự báo là bài toán toán được nhiều nhà nghiên cứu trong cộng đồng phát hiện bất thường quan tâm. Bởi vì đây là một cách tiếp cận mới, tận dụng được sự bùng nổ của các mạng nơ-ron học sâu. Phương pháp này gồm hai công tác chính là dự báo dữ liệu và phát hiện bất thường. Các bộ dữ liệu dùng để phát hiện bất thường là các bộ dữ liệu được các nhà nghiên cứu sử dụng nhiều trong công tác phát hiện bất thường, các bộ dữ liệu này là một chuỗi thời gian, có tương quan với nhau. Để có thể dự báo chính xác với dữ liệu chuỗi thời gian có tương quan với nhau, nghiên cứu này đề xuất mô hình sử dụng mạng nơ-ron học sâu LSTM xếp chồng. Ngoài ra để nâng cao khả năng dự báo, chúng tôi cũng áp dụng kỹ thuật dự báo nhiều bước cho mô hình đề xuất. Sai số dự báo sẽ được sử dụng cho công tác phát hiện bất thường. Để đánh giá mô hình, chúng tôi so sánh với giải thuật HOTSAX về kết quả phát hiện bất thường và thời gian thực thi. Kết quả thực nghiệm trên 08 bộ dữ liệu cho thấy mô hình đề xuất của chúng tôi đã khắc phục được hạn chế của việc dựa vào kích thước cửa sổ trượt trong giải thuật HOTSAX, đồng thời cũng khẳng định tiềm năng của phương pháp phát hiện bất thường bằng dự báo.


\clearpage